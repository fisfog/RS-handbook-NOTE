% !TEX program = xelatex


\documentclass{article} 
\usepackage{xltxtra} % 这个包是为了打印\XeLaTeX 的Logo。 
\usepackage{xeCJK} % 这个包可以指定中文字体。
\setCJKmainfont[Mapping=tex-text,BoldFont=SimHei]{SimSun}
\setCJKsansfont{SimHei}
\setCJKmonofont[Scale=.85]{FandolFang}
\setmainfont{Palatino Linotype}
\setmonofont{Consolas}

\usepackage{url}
\usepackage{geometry}
\geometry{left=3cm,right=3cm}

\usepackage{multirow}
\usepackage{amsfonts}

\newcommand{\e}[1]{$ \times 10^{#1}$}
\renewcommand{\figurename}{图}
\renewcommand{\tablename}{表}
\renewcommand{\today}{\number\year 年 \number\month 月 \number\day 日}


\begin{document}
 \title{Recommder System Handbook学习笔记\ 第一部分,章节6 发展中的基于约束推荐}
 \author{littlekideee}
 \maketitle

 \section{介绍}
 传统的推荐方法很好地适用于想图书、电影或新闻等产品的推荐,。但当推荐如汽车、计算机、公寓或金融服务的时候这些方法就不是一个很好的选择了。

 基于知识的推荐技术帮助解决这些挑战。它利用明确的用户需求和关于基础产品领域的深度知识来计算推荐。难点在于领域知识的获取瓶颈,而且知识工程师需要大量的工作去将知识转化到形式化地表示上。

 现有两种基于知识的推荐:1.基于案例(case-based)2.基于约束(constraint-based)。就使用的知识而言两者是相似的:用户需求

\end{document}